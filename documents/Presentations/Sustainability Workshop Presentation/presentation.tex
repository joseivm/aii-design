\documentclass{beamer}
\usepackage[utf8]{inputenc}
\usepackage[english]{babel}
\usepackage{hyperref}
\usepackage{graphicx}
\graphicspath{{./Figures/}}
\usepackage{wrapfig}
\usepackage{subcaption}
\usepackage{booktabs}
% \usepackage[square,numbers]{natbib}
%\bibliographystyle{unsrtnat}
\makeatletter
\let\@@magyar@captionfix\relax
\makeatother

\newtheorem{defn}[theorem]{Definition}
\DeclareMathOperator*{\argmax}{arg\,max}
\DeclareMathOperator*{\argmin}{arg\,min}
\hypersetup{
    colorlinks=true,
    linkcolor=blue,
    filecolor=blue,      
    urlcolor=blue,
    citecolor=black
}

\usetheme{Boadilla}
\usecolortheme{beaver}

 \usepackage[
    backend=biber,
    style=authoryear,url=false
  ]{biblatex}

\addbibresource{../../main.bib}

\title[Agricultural Index Insurance]{Agricultural Index Insurance: An Optimization Approach}
\author{José I. Velarde Morales \and Linwei Xin}
\institute[Chicago Booth]
{

  University of Chicago\\
  Booth School of Business

}


  


\begin{document}
\frame{\titlepage}
\section{Introduction}

\subsection{The Problem of Agricultural Risk}
\begin{frame}{The Problem of Agricultural Risk}
\begin{itemize}
    \setlength\itemsep{2em}
    \item Farmers face a lot of risk, and the lack of risk management tools forces them to use coping strategies that hurt their long term welfare.
   
    \item Traditional insurance is prohibitively costly in most developing countries due to lack of data and high verification costs.

    \item Moral hazard, adverse selection, and the presence of large covariate shocks make the problem of agricultural insurance especially hard. 
\end{itemize}
\end{frame}

\subsection{A Proposed Solution: Index Insurance}
\begin{frame}{A Proposed Solution: Index Insurance}
\begin{itemize}
   \setlength\itemsep{1.5em}
    \item In index insurance, an index (or statistic) is created using easily observable quantities (e.g. rainfall), and it is used to determine whether the insured party suffered an adverse event. 
    \item If the index falls below a pre-determined threshold, the insurance company automatically issues out payments to the insured. 
    \item This allows the insurance company to reduce verification costs.
\end{itemize}
\end{frame}

\begin{frame}{Index Insurance in Practice}
\begin{itemize}
    \setlength\itemsep{1.5em}
    \item Since it was first proposed, index insurance programs have been implemented in many countries including India, Mexico, Tanzania, Malawi, Kenya, and many others (\cite{jensen2017agricultural}). 
    
    \item Today, tens of millions of farmers worldwide are covered by index insurance programs (\cite{greatrex2015scaling}). 
    
    \item However, in most of these cases, the insurance has to be heavily subsidized by governments due to high cost and low demand (\cite{greatrex2015scaling}). 
\end{itemize}
\end{frame}

\subsection{Project Overview}
\begin{frame}{Project Overview}
 \begin{itemize}
    \setlength\itemsep{1em}
    \item Traditionally, the contract for each insured zone is designed independently of all other zones.
     \item The goal of this project is to make insurance less costly by improving the design of the insurance contracts. 
    \item Our method simultaneously determines the contract parameters for different areas, while taking into account the correlation between the areas. 
 \end{itemize}
\end{frame}





\section{Optimization Approach}
\begin{frame}{Model Overview}
    \begin{itemize}
       \setlength\itemsep{1.5em}
        \item We conducted interviews with researchers and practitioners that had implemented index insurance programs in several countries (Malawi, Kenya, Senegal, Thailand, among others) to learn more about the context. 
        \item \textbf{Objective:} minimize risk faced by farmers 
        \item \textbf{Constraints:} Budget constraints and price constraints
        \item \textbf{Cost to insurer:} sum of payouts plus cost of required capital: $C(\theta_z) = \sum_z P_z(\theta_z) + c_k K$
    \end{itemize}    
    \end{frame}

% \subsection{CVaR Model}

% \begin{frame}{Idealized CVaR Model}
%     \begin{itemize}
%         \item \textbf{Objective:} conditional value at risk of the farmers' loss net of insurance.
%         \item  \textbf{Constraint 1:} budget constraint
%         \item \textbf{Constraint 2:} definition of required capital
%         \item \textbf{Constraint 3:} structure of contract
%     \end{itemize}

%     \begin{align}
%         \min_{a,b,K,\pi} \max_z &\quad CVaR_{1-\epsilon}\left (\ell_z  + \pi_z - I_z(\theta_z)\right ) \notag\\
%         \text{s.t.   } \quad & \mathbb{E}\left [ \sum_z I_z(\theta_z) \right ] +  c_{\kappa} K \leq B \\
%         K &= CVaR_{1-\epsilon_K} \left( \sum_z s_z I_z(\theta_z) \right ) - \mathbb{E}\left [ \sum_z s_zI_z(\theta_z) \right ]\\
%         &I_z(\theta_z) = \min \left \{ \max \left \{0,a_z\hat{\ell_z}(\theta_z) + b_z \right \},1 \right \}
%       \end{align}
     
%     \end{frame}


\section{Evaluation}
\subsection{Baseline Method}
\begin{frame}{Evaluation Method}
    \begin{itemize}
        \setlength\itemsep{2em}
        \item \textbf{Baseline Method:} We compare our method to the method developed in \cite{chantarat2013designing}. This is the method that was used to design Kenya's index insurance program, and is what is most commonly used in academic publications (see \cite{flatnes2018improving}; \cite{jensen2019does}). 
        \item \textbf{Kenya Household Survey Data:} We used a household survey of Kenyan pastoralists. This survey tracked monthly livestock levels and losses for 900 households between 2010-2013. 
    \end{itemize}
\end{frame}

\begin{frame}{Evaluation Procedure}
\begin{enumerate}
\setlength\itemsep{2em}
    \item Split data into training and test sets
    \item Use training set to design insurance contracts using both methods
    \item Apply insurance contracts designed by the two methods to farmers in the tests set and compare outcomes. 
\end{enumerate}
Performance Metrics: Conditional Value at Risk $(CVaR)$, Value at Risk $(VaR)$, semi-variance.
\end{frame}

\subsection{Results}
\begin{frame}{Results}
 The insurance contracts developed by our method are $20\%$ less costly than the baseline with this dataset, while offering comparable coverage. 

 \begin{table}[]
\small
    \centering
\begin{tabular}{lrrrr}
\toprule
   Model &  Max CVaR &  Max VaR &  Max SemiVar &   Average Cost \\
\midrule
Baseline &      0.69 &     0.52 &         0.22 &             \textbf{5263.64} \\
     Opt &      0.65 &     0.53 &         0.22 &            \textbf{3794.80} \\
\bottomrule
\end{tabular}
    \caption{Results using Kenya household data}
    
\end{table}
    
\end{frame}

\begin{frame}{Summary of Results}
    \begin{itemize}
        \setlength\itemsep{1.5em}
            \item Our method offers comparable coverage to the status quo as measured by $CVaR, VaR,$ and semi-variance
            \item Contracts designed by our method are over $15\%$ less costly on average.
            \item Our method reduces costs associated with required capital. 
            \item Our method is more robust to misspecification of prediction model. 
        \end{itemize}
    
\end{frame}

% \begin{frame}{Why our method outperforms the baseline}
% \begin{itemize}
% \setlength\itemsep{2em}
%     \item Adjusts payout strategy based on the correlation between the insured zones, making it less risky for the insurer. 
%     \item Our method is more robust to misspecification of prediction model. 
% \end{itemize}
% \end{frame}

\begin{frame}{Summary}
    \begin{itemize}
        \setlength\itemsep{1.5em}
        \item We make index insurance more cost effective by improving the design of the insurance contracts. 
        \item Our method simultaneously designs the contracts for all insured zones, taking into account correlation between zones.
        \item Our method is able to outperform baseline because it is better at managing risk and because it is more robust to errors in the prediction model. 
    \end{itemize}
    
\end{frame}







\begin{frame}[noframenumbering, plain]{References}
\printbibliography
\end{frame}




\end{document}