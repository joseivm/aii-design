\documentclass[11pt]{article}
\usepackage[margin=1in]{geometry}
\usepackage{amsmath,amsthm,amssymb,bbm}
\usepackage{float}
\usepackage{hyperref}
\usepackage{booktabs}
\usepackage{placeins}
\usepackage{graphicx}
\usepackage{authblk}
\usepackage[
backend=biber,
style=bwl-FU,
sorting=ynt
]{biblatex}
\addbibresource{../main.bib}

\DeclareMathOperator*{\argmax}{arg\,max}
\DeclareMathOperator*{\argmin}{arg\,min}

\title{Optimal Index Insurance Contract Design}
\author{José I. Velarde Morales}
\author{Linwei Xin}
\affil{University of Chicago - Booth School of Business}


\begin{document}
\maketitle

% \section{Background}
% Farmers face large amounts of risk. This risk can come from multiple sources. There are risks coming from the weather, such as droughts or floods. There is also risk associated with market forces: the price of the good being produced can vary dramatically year to year. Many of the risk management tools available to farmers in wealthy countries (e.g. insurance or futures), are not available to farmers in poorer countries. As a result, when negative shocks occur, farmers are often forced to rely on coping strategies that are detrimental to their long term welfare. For example, they might be forced to sell productive assets or withdraw their children from school.
% Traditional insurance is too costly to implement in many poor countries. One of the main drivers of this is the high costs of verification. In contrast to rich countries, where agriculture tends to be more concentrated, in poor countries agriculture is often characterized by smaller farmers spread out over large regions. This increases the number of claims insurers have to verify, and it makes it more costly to verify each claim. The presence of large covariate shocks translates to more risk for the insurer. 
% \section{Option 1}
Uninsured risk limits agricultural productivity and growth in developing countries. Since nearly two thirds of the world's poor are employed in agriculture, addressing this problem could have significant welfare implications. Agricultural insurance is, even in the best circumstances, a hard problem. Many of the features one would want (independent units, uncorrelated risk, etc) are missing in this context. When considering insurance in developing countries, the problem becomes even harder because of verification costs. Traditionally, whenever an adverse event happens, the insured party contacts the insurer, and the insurer verifies the claim and issues a payout. However, agriculture in developing countries is often characterized by many small farmers spread out over hard to reach regions. This makes verification prohibitively costly. Additionally, the presence of correlated risks makes insurance more expensive because it makes large payouts more likely. Intuitively, if one farmer is affected by a drought, it is likely that other farmers were also affected. If large payouts are more likely, the insurer must have larger reserves in order to maintain solvency. 

Researchers developed index insurance as a less costly way to offer insurance in developing countries. In index insurance, an index (or statistic) is created using easily observable quantities, and it is used to determine whether the insured party suffered an adverse event. In the past, indices have been constructed using rainfall, weather, and satellite images. If the index falls below a pre-determined threshold, the insurance company automatically issues out payments to the insured. This allows the insurance company to circumvent the issue of verification, moral hazard, and adverse selection, since the actions of individual farmers cannot affect the index. Even though index insurance has proved to be a less costly way of providing insurance for small farmers, it has been difficult to scale up. There are several problems with index insurance. One of the main problems is low take up: farmers are often unwilling to purchase the insurance at market prices. Another problem, as previously mentioned, is the cost. The goal of this project is to make insurance more cost effective by improving the design of the contracts. %We develop a method to simultaneously design the contracts of all the insured zones while taking into account the correlations between the zones. This allows us to make better tradeoffs between coverage and the cost capital. 

We conducted interviews with researchers and practitioners to learn more about the context. Based on these interviews we formulated the problem of designing insurance contracts as an optimization program. The objective in our program was to minimize the risk faced by farmers subject to a constraint on the price of the product. Our method simultaneously designs the contracts for all insured zones, taking into account the correlations between the zones. We compare our method to the method developed in \cite{chantarat2013designing}, which is what's most commonly used in the academic literature and what is used in Kenya's Index Insurance program. We compare the two methods using both real and synthetic data, and find that our method is able to provide better coverage at the same cost as the baseline method. 

% \section{Option 2}
% We develop an optimization based approach to designing index insurance contracts that improves coverage and is less risky for the insurer. Index insurance is a popular way of providing agricultural insurance in developing countries. Index insurance programs have been implemented in a variety of countries (e.g. India, Mexico, Tanzania) and it is estimated that tens of millions of farmers worlwide are covered by such products \cite{greatrex2015scaling}. In index insurance, an index (or statistic) is created using easily observable quantities, and it is used to determine whether the insured party suffered an adverse event. In the past, indices have been constructed using rainfall, weather, and satellite images. If the index falls below a pre-determined threshold, the insurance company automatically issues out payments to the insured. This allows the insurance company to circumvent the issue of verification, moral hazard, and adverse selection, since the actions of individual farmers cannot affect the index. Even though index insurance has proved to be a less costly way of providing insurance for small farmers, it has been difficult to scale up. There are several problems with index insurance. One of the main problems is low take up: farmers are often unwilling to purchase the insurance at market prices. Another problem, as previously mentioned, is the cost. The presence of correlated risks makes insurance in these scenarios more expensive because it makes large payouts more likely. Intuitively, if one farmer is affected by a drought, it is likely that other farmers were also affected. If large payouts are more likely, the insurer must have larger reserves in order to maintain solvency. We developed a method to simultaneously design the contracts of all the insured zones while taking into account the correlations between the zones. This allows us to make better tradeoffs between coverage and the cost capital.

% We conducted interviews with researchers and practitioners to learn about the context and inform the design of the program. Our program's objective is to minimize risk faced by farmers subject to a constraint on the price of the product. We use the Conditional Value at Risk $(CVaR)$ as our measure of risk, and derive a convex approximation to the problem. We evaluate our method by comparing its performance with the method developed by \cite{chantarat2013designing}. This method is the standard method used in academic publications describing the design of index insurance contracts (see \cite{flatnes2018improving}; \cite{jensen2019does}). It is also what was used to design Kenya's Index Based Livestock Insurance (IBLI) program. We first compare the two methods using synthetic data. We compare the performance of the two methods under different scenarios with varying degrees of correlation between the insured zones. We also compare the how the two methods are affected by the quality of the underlying prediction model. We find that our method either matches or outperforms the baseline in all scenarios tested. We also evaluate the two methods using data from Kenya's IBLI program and find similar results. Our method provides comparable or better coverage at the same cost, and is generally less risky for the insurer. This holds even when the unerlying prediction model has poor performance. 

% We evaluate how the performance of the two methods is affected by the correlation between the insured zones and the quality of the underlying prediction model. We then evaluate the two methods using data from Kenya's IBLI program. We find that our method either matches or outperforms the baseline in all scenarios. Our method provides comparable or better coverage at the same cost, and is generally less risky for the insurer. This holds even when the unerlying prediction model has poor performance. 

\end{document}