\documentclass[11pt]{article}
\usepackage[margin=1in]{geometry}
\usepackage{amsmath,amssymb}
\usepackage[
backend=biber,
style=bwl-FU,
sorting=ynt
]{biblatex}
\addbibresource{main.bib}

\title{An Optimization Based Approach to the Design of Agricultural Index Insurance}
\author{José I. Velarde Morales}

\begin{document}
\maketitle
\section{Introduction}
Lack of access to credit and insurance is often cited as a significant factor hindering agricultural productivity in developing countries. Nearly two thirds of the world's poor are employed in agriculture, and addressing this problem could have significant welfare implications. Agricultural insurance is, even in the best circumstances, a hard problem. Many of the features one would want (independent units, uncorrelated risk, etc) are missing in this context. When considering insurance in developing countries, the problem becomes even harder because of verification costs. Traditionally, whenever an adverse event happens, the insured party contacts the insurance company, and the insurance company verifies the claim and issues a payout. However, agriculture in developing countries is often characterized by many small farmers spread out over hard to reach regions. This makes verification prohibitively costly. 

Researchers developed index insurance as a less costly way to offer insurance in developing countries. In index insurance, and index (or statistic) is created using easily observable quantities, and it is used to determine whether the insured party suffered an adverse event. In the past, indices have been constructed using rainfall, weather, and satellite images. If the index falls below a pre-determined threshold, the insurance company automatically issues out payments to the insured. This allows the insurance company to circumvent the issue of verification, moral hazard, and adverse selection, since the actions of individual farmers can't affect the index. Even though index insurance provided a less costly way of providing insurance for small farmers, it has been difficult to scale up. There are several problems with index insurance. One of the main problems is low take up: farmers are often unwilling to purchase the insurance at market prices. Another problem, as previously mentioned, is the cost. The purpose of this project is to make this insurance less costly by improving the design of insurance contracts. In the following section we describe the methods that are currently employed in the design of index insurance. In the last section we describe our proposed improvements. 

\section{Status Quo}
  \subsection{Brief Lit Review}
    There are many studies that evaluate how index insurance impacts the behavior of farmers (see, for example \cite{karlan2014agricultural}). However, there has been relatively little research concerning the design of index insurance. In \cite{chantarat2013designing}, the authors describe the design of an index insurance for pastoralists in Northern Kenya. The authors use a satellite based index, the Normalized Difference Vegetation Index, to predict herd mortality. They use a statistical clustering method to assign each $8km^2$ block to a larger region. Then, for each region, they specify a piecewise linear contract based on the predicted herd mortality. In \cite{jensen2019does}, the authors compare the welfare implications of using different satellite based indices for insuring pastoralists against drought. In this study, the regions are taken as given, and they also use piecewise linear contracts. As to the setting of strike values, the authors offer the following: "Similar to IBLI (Index Based Livestock Insurance) policies being sold, the strike for each index product was set at each index's within-index-unit 20th percentile". Using these contracts, the authors then compared how the farmers would have fared under each of the different insurance contracts. 
    
    In \cite{flatnes2018improving}, the authors propose augmenting a traditional index insurance contract with the option for an audit. In this augmented contract, the insured farmer has the option to request an audit if they believe a payout should have been issued but wasn't. In this case, the insurance company would check the farmer's claim and issue a payment if the claim is correct. If the claim is correct, the insurance company would cover the cost of the audit, if it was incorrect, the farmer would have to cover the cost of the audit. In this study, the insurance zones are based on observed clustering of fields. "The main crops grown are paddy and maize, and paddy fields are clustered together... The paddy clusters are also used as the basis for the insurance zones". Here, the insurance contract pays out the losses predicted by the index. There are no trigger values set or piecewise linear contracts. 

    In general, there does not seem to be a standard methodology for developing index insurance products. As stated in \cite{world2011weather}, "The reader should be aware that there is no single methodology in this field ... [this paper] describes an approach that has been used in a number of index pilot activities undertaken by the World Bank and its parners." 

  \subsection{Context}
    Index insurance is offered by regional insurance companies or by public-private partnerships. Since insurance companies often don't have the technical knowledge to develop index insurance by themselves, they often rely on the help of international research institutes, such as the Columbia University International Research Institute for Climate and Society, to design, underwrite and price these insurance products. 

  
  \subsection{Current Methodology}
    In general, weather index insurance contracts specify a coverage period which is split up into different phases. Each of these phases corresponds to a different stage of growth in the crop cycle. The stages are establishment/vegetative, flowering/reproductive, and the ripening stage. Each stage has different climate requirements, and thus a different payout function is calculated for each phase. The payout function for each phase has three parameters: 
    \begin{itemize}
      \item \textbf{trigger:} this is a threshold value for the index. If the index falls below this value, then a payout is automatically issued. 
      \item \textbf{exit:} this is the value at which the maximum payment issued. 
      \item \textbf{maximum payout}: this is the amount that is paid if the index is below the exit level. 
    \end{itemize}

    As described in \cite{world2011weather}, the current method for doing things is to optimize the trigger values for each region independently. A numerical optimization method is used to decide the trigger values for each of the phases. The objective in this optimization is to minimize the variance of farmer losses subject to a price constraint. In the optimization, the loss is predicted using a WRSI (Water Requirement Satisfaction Index) model. The optimization is nonconvex, and the authors use the Nelder Mead algorithm to solve the program. Since the optimization method is not guaranteed to find a global optimum, is mostly used to discover local optima close to an initial guess. Contract parameters are often adjusted after the optimization to better address the needs of different stakeholders.

    The assignment of smaller geographical units to insurance zones is treated as exogenously given, or is decided using statistical clustering methods.

\section{Research Directions}
  We believe that the current methodology can be improved upon in four ways: 
  \begin{enumerate}
    \item We could simultaneously optimize the contracts for all regions being considered. Jointly deciding the contract parameters for all regions could help insurers better manage their risk and allow them to offer lower premiums. We plan to model the uncertainty in this problem using distributionally robust optimization (see \cite{delage2010distributionally} and \cite{bertsimas2011theory}). Robust optimization has recently been used to model uncertainty in contract design problems (see \cite{yu2020robust}), and we hope to extend that approach to this setting. 
    \item Using an approach similar to \cite{papalexopoulos2022ethics}, we can use a machine learning model to create a tractable approximation to the WRSI model used to predict farmer losses. This could allow us to add complexities to the optimization problem that could lead to efficiency gains (e.g. taking into account correlations between regions). 
    \item For each region, we could specify different insurance contracts catered to different risk preferences. For example, having a contract for very risk averse farmers, and another one for farmers that are more risk neutral. This could help in increasing take up, since risk preferences will presumably vary across farmers. 
    \item We could use an optimization based approach to assign smaller geographical units to insurance zones. Optimizing the regional assignment has led to efficiency gains in organ transplantation \cite{gentry2015gerrymandering}, and could possibly allow insurers to manage risk better. 
  \end{enumerate}


\printbibliography



\end{document}