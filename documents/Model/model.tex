\documentclass[11pt]{article}
\usepackage[margin=1in]{geometry}
\usepackage{amsmath,amssymb,bbm}
\usepackage{float,color}
\usepackage{hyperref}
\usepackage{booktabs}
\usepackage{placeins}
\usepackage{graphicx}
\usepackage[
backend=biber,
style=bwl-FU,
sorting=ynt
]{biblatex}
\addbibresource{../main.bib}

\newtheorem{fact}{Fact}
\newtheorem{lemma}{Lemma}
\newtheorem{theorem}[lemma]{Theorem}
\newtheorem{defn}[lemma]{Definition}


\title{Agricultural Index Insurance Model}
\author{José I. Velarde Morales}

\begin{document}
\maketitle

\section{Index Insurance Definition and Parameters}
    Index insurance generally involves an easily observable signal, $\theta$, that is used to predict the loss, $\hat{\ell}(\theta)$, of some agricultural product. For example, $\theta$ could be rainfall, and $\hat{\ell}(\theta)$ could be livestock mortality. Index insurance contracts normally have the form: $I(\hat{l}(\theta)) = \min \left \{\max \left \{0,a\hat{\ell}(\theta) + b \right \}, P \right \}$, where $P$ is the maximum payout, and $a,b$ are the contract parameters. For ease of notation, we will use $I(\theta)$ instead of $I(\hat{\ell}(\theta))$. The expected cost, $C(I(\theta))$ of an insurance contract, $I(\theta)$ for an insurer in a single period is: $C(I(\theta)) = \mathbb{E}[I(\theta)] + c_{\kappa} K(I(\theta))$, where $c_{\kappa}$ is the cost of holding capital, and $K$ is the amount of capital required to insure the contract. $K$ is set by regulators, and is meant to ensure that insurers have enough capital to fulfill their contractual obligations with high probability. One commonly used formula for $K$ is $K(I(\theta)) = CVaR_{1-\epsilon_P}\left ( I(\hat{l}(\theta)) \right ) - \mathbb{E}[I(\theta)]$ (\cite{mapfumo2017risk}). $\epsilon_P$ is set by regulators, and commonly used values are $\epsilon_P = 0.01$ or $\epsilon_P = 0.05$. 

\section{CVaR Model}
  \subsection{Motivation}
    We want a model that minimizes the probability that wealth drops below a certain threshold, subject to a budget constraint. However, probabilistic objectives are generally non-convex. The probability that wealth drops below a certain threshold is a measure of the risk farmers face; we want a convex measure that will capture this risk. We use the Conditional Value at Risk $CVaR$ of the loss net of insurance as our measure of risk. 

  % \paragraph{Conditional Value at Risk} 
  \begin{defn}
    For a random variable $z$, representing loss, the $(1-\epsilon)$ Value at Risk $(VaR)$ is given by 
    \begin{align*}
      VaR_{1-\epsilon}(z) := \inf \left \{ t : P(z \leq t) \geq 1-\epsilon \right \}
    \end{align*}
  \end{defn}

  \begin{defn}
    For a random variable $z$, representing loss, the $(1-\epsilon)$ Conditional Value at Risk $(CVaR)$ is given by 
    \begin{align*}
      CVaR_{1-\epsilon}(z) := \mathbb{E}\left [z | z \geq VaR_{1-\epsilon}(z) \right ]
    \end{align*}
  \end{defn}

  Intuitively, the Conditional Value at Risk is the expected value of a loss given that it is above a certain threshold. It can also be thought of as measuring the worst outcomes. By minimizing the $CVaR$ of the net loss, we are focusing on improving farmers' wealth in the worst case scenarios, which is a key purpose of insurance. $CVaR$ has been extensively studied in the academic literature, and is also popular in practice (\cite{rockafellar2000optimization},\cite{rockafellar2002conditional},\cite{artzner1999coherent}). It also has the advantage of being convex, and thus amenable to optimization. This leads us to the following model: 

  \begin{align}
  \min_{a,b,\pi, K}  & \quad CVaR_{1-\epsilon}\left ( \ell + \pi - I(\theta) \right)\\
  \text{s.t.   }I(\theta) &= \min \{ (a\hat{\ell}(\theta) + b)^+,1\} \\
   \pi &= \mathbb{E}\left [ I(\theta) \right ] + c_{\kappa} K \\
  K &=  CVaR_{1-\epsilon}\left ( I(\theta)  - \mathbb{E}[I(\theta)] \right ) \label{cons-budget} \\
  \pi &\leq \overline{\pi}
  \end{align}

  Our objective is the conditional value at risk of the farmers' loss net of insurance. The first constraint specifies the piecewise linear structure of the contract, and the second constraint is the budget constraint. Recall that the total cost of insurance consists of payouts plus the cost of capital. The last constraint is just the definition of the required capital. 
  
  \subsection{Changes to constraints}
  To make the program convex, we replace $I(\theta)$ with conservative approximations where necessary, making sure that by doing so we still get a feasible solution. We use the following approximations of $I(\theta)$ to make the problem convex: 
\begin{align*}
    \overline{I(\theta)} &\triangleq \max \left \{ 0,a\hat{\ell}(\theta) + b\right \} \\
    \underline{I(\theta)} &\triangleq \min \{ a\hat{\ell}(\theta) + b,1 \}\\
    &\implies \\
    \underline{I(\theta)} &\leq I(\theta) \leq \overline{I(\theta)}
\end{align*}
  
  Note that $\overline{I(\theta)} \geq I(\theta)$ and $\overline{I(\theta)}$ is convex. Conversely, $\underline{I(\theta)} \leq I(\theta)$ and $\underline{I(\theta)}$ is concave. We replace $I(\theta)$ with either $\overline{I(\theta)}$ or $\underline{I(\theta)}$ where necessary to obtain conservative and convex approximations. We replace $I(\theta)$ in the objective with $\underline{I(\theta)}$. This will give us an lower bound on the performance of our contracts, since $\ell - \underline{I(\theta)} \geq  \ell - I(\theta)$ . We replace $I(\hat{\ell}(\theta))$ in the budget constraint with $\overline{I(\theta)}$. This is a conservative approximation of the constraint, since $I(\theta)  \leq \overline{I(\theta)}$. We replace $I(\theta)$ in constraint \ref{cons-budget} with $\overline{I(\theta)}$ or $\underline{I(\theta)}$ depending on the sign to keep convexity. This also yields a conservative approximation since: $CVaR_{1-\epsilon}\left ( I(\theta) \right )  - \mathbb{E}[I(\theta)]  \leq CVaR_{1-\epsilon}\left ( \overline{I(\theta)} \right )  - \mathbb{E}[ \underline{I(\theta) }]$. Finally, we set $\pi = \mathbb{E}\left [ \overline{ I(\theta)} \right ] + c_{\kappa} K$.
  
  \subsection{Single Zone Model}
  Our single zone model minimizes the conditional value at risk of farmers' net loss (i.e. loss net of insurance), subject to an overall budget constraint. The total cost of the insurance consists of payouts and of costs associated with holding capital. We first describe the model parameters, and then describe the model. The model will output values for $a$ and $b$, and the final insurance contracts will be $I(\theta) = \min \left \{\max \left \{0,a\hat{\ell}(\theta) + b \right \}, 1 \right \}$.
  \paragraph*{Model Parameters}
  \begin{itemize}
    \item $\epsilon$: This is the $\epsilon$ used for the $CVaR$ objective.  $\epsilon = 0.1$ means that our objective is $E[\ell - I(\theta)|\ell -I(\theta) \geq VaR_{1-0.1}\left ( \ell - I(\theta) \right )]$. 
    \item $\epsilon_K$: This is the epsilon used in the formula for required capital. Recall that the required capital $K = CVaR_{1-\epsilon_K}(I(\theta)) - E[I(\theta)]$.
    \item $\overline{\pi}$: This is the budget constraint for the total cost of the insurance. 
    \item $s$: This is the total insured amount.
    \item $c_{\kappa}$: This is the cost of capital. 
\end{itemize}
  
  \paragraph*{Model}
  In the model below, our objective is the conditional value at risk of the farmer's loss net of insurance. The first constraint is the budget constraint and the last constraint is the definition of the required capital. $\ell, \pi, I(\theta)$ are in terms of rates. So, for example, $\ell$ represents the share of the product that was lost. $\pi$ is also expressed as a share of the total insured amount. And similarly, $I(\theta)$ is the share of the insured amount that is paid out. 
  \begin{align}
    \min_{a,b,K,\pi} &\ CVaR_{1-\epsilon}\left(s(\ell + \pi  -  \underline{I(\theta)} )\right)\\
    \text{s.t.   } & \pi = \mathbb{E} \left [ \overline{I(\theta)} \right ] + \frac{1}{s}c_k K\\
    K &= CVaR_{1-\epsilon_P} \left( s\overline{I(\theta)} \} \right) - \mathbb{E}\left [ s\underline{I(\theta)} \right ]\\
    \overline{I(\theta)} &= \max \left \{0,a\hat{\ell}(\theta) + b \right \}\\
    \underline{I(\theta)} &= \min \left \{a\hat{\ell}(\theta)+b,1 \right \}\\
    \pi &\leq \overline{\pi}
\end{align}

  We reformulated the problem as a linear program  using the results from \cite{rockafellar2000optimization}. In the model below, $p^j$ is the probability of event $j$, and $j$ indexes the possible realizations of $\theta, \ell$. $N$ is the total number of samples. 
  
  \begin{align}
      \min_{a,b,\gamma,\gamma_K,\alpha,t,t_K} &\quad t + \frac{1}{\epsilon}\sum_j p^j \gamma^j\\
      \text{s.t.   } \gamma^j &\geq s(\ell^j + \pi - \omega^j)  - t, \forall j\\
      \gamma^j &\geq 0, \forall j \\
        \pi &= \frac{1}{N}\sum_j \alpha^j + \frac{1}{s} c_{\kappa} K\\
        t_K &+ \frac{1}{\epsilon_K} \sum_j p^j \gamma_K^j \leq K+ \frac{1}{N}\sum_j \omega^j \\
        \gamma_K^j &\geq s\alpha^j -t_K, \forall j \\
        \gamma_K^j &\geq 0, \forall j\\
        \alpha^j &\geq a \hat{\ell}(\theta^j) + b, \forall j\\
        \alpha^j &\geq 0, \forall j\\
        \omega^j &\leq a \hat{\ell}(\theta^j) + b, \forall j\\
        \omega^j &\leq 1, \forall j\\
        \pi &\leq \overline{\pi}
  \end{align}
    
  \subsection{Multiple Zone Model}
  The multiple zone model is very similar to the single zone model. In the objective, we minimize the maximum conditional value at risk of  farmers' net loss across all zones, $z$. We minimize the maximum $CVaR$ across all zones to avoid situations where one zone has a contract that is significantly worse than other zones. The other change is that the budget constraint includes the payouts of all zones, and the required capital is determined using the sum of payouts across all zones.
    \paragraph*{Model Parameters}
    \begin{itemize}
        \item $\epsilon$: This is the $\epsilon$ used for the $CVaR$ objective.  $\epsilon = 0.1$ means that our objective is $E[\ell - I(\theta)|\ell -I(\theta) \geq VaR_{1-0.1}\left ( \ell - I(\theta) \right )]$.  
        \item $\epsilon_K$: This is the epsilon used in the formula for required capital. Recall that the required capital $K(I(\theta)) = CVaR_{1-\epsilon_K}(I(\theta)) - E[I(\theta)]$. 
        \item $c_{\kappa}$: cost of capital. 
        \item $s_z$: total insured amount for zone $z$.
    \end{itemize}

    \paragraph*{Model}
    In the model below, our objective is the maximum conditional value at risk of the net loss across all zones. The second constraint is the budget constraint, which now includes the sum of payouts across all zones. The formula for required capital was also changed to include the sum of payouts across all zones. 
    \begin{align}
      \min_{a,b,K,\pi} \max_z &\quad CVaR_{1-\epsilon}\left (s_z \left (\ell_z  + \pi_z - \underline{I_z(\theta_z)}\right ) \right )\\
      \text{s.t.   } & \pi_z  = \mathbb{E}\left [ \overline{I_z(\theta_z)} \right ] + \frac{1}{\sum_z s_z} c_{\kappa} K \\
      K &= CVaR_{1-\epsilon_K} \left( \sum_z s_z\overline{I_z(\theta_z)} \right ) - \mathbb{E}\left [ \sum_z s_z\underline{I_z(\theta_z)} \right ]\\
      &\overline{I_z(\theta_z)} = \max \left \{0,a_z\hat{\ell_z}(\theta_z) + b_z \right \}\\
      &\underline{I_z(\theta_z)} = \min \left \{a_z\hat{\ell_z}(\theta_z)+b_z,1 \right \}\\
      &\pi_z \leq \overline{\pi_z}
    \end{align}

    We reformulated the problem as a linear program  using the results from \cite{rockafellar2000optimization}. In the model below, $p^j$ is the probability of event $j$, and $j$ indexes the possible realizations of $\theta, \ell$. $N$ is the total number of samples. 

    \begin{align}
      \min_{a,b,\alpha,\omega,\gamma,t,m,K,\pi} \quad & m\\
      \text{s.t.} \quad t_z &+ \frac{1}{\epsilon} \sum_j p^j \gamma_z^j \leq m, \forall z\\
      \gamma_z^j &\geq s_z \left (\ell^j + \pi_z - \omega^j_z \right ) -t_z, \forall j, \forall z \\
      \gamma_z^j &\geq 0, \forall j, \forall z\\
      \pi_z &= \frac{1}{N} \sum_j \alpha^j_z + \frac{1}{\sum_z s_z} c_k K\\
      t_K &+ \frac{1}{\epsilon_K} \sum_j p^j \gamma_K^j \leq K+ \frac{1}{N}\sum_j \sum_z s_z \omega^j_z\\
      \gamma_K^j &\geq \sum_z s_z \alpha^j_z -t_K, \forall j \\
      \gamma_K^j &\geq 0, \forall j\\
      \alpha^j_z &\geq a_z \hat{\ell_z}(\theta^j_z) + b_z, \forall j, \forall z\\
      \alpha^j_z &\geq 0, \forall j, \forall z\\
      \omega^j_z &\leq a_z \hat{\ell_z}(\theta^j_z) + b_z, \forall j, \forall z\\
      \omega^j_z &\leq 1, \forall j, \forall z\\
      \pi_z &\leq \overline{\pi_z}, \forall z
    \end{align}


\end{document}