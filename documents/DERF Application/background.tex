\documentclass[12pt]{article}
\usepackage{times}
\usepackage[margin=1in]{geometry}
\usepackage{amsmath,amsthm,amssymb,bbm}
\usepackage{float}
\usepackage{hyperref}
\usepackage{booktabs}
\usepackage{placeins}
\usepackage{graphicx}
\usepackage{setspace} \doublespacing
\usepackage[
backend=biber,
style=bwl-FU,
sorting=ynt
]{biblatex}
\addbibresource{../main.bib}

\usepackage{titling}

\setlength{\droptitle}{-8em}

\title{Narrative of Research Background\vspace{-10ex}}
\date{}
\begin{document}
% \maketitle
The goal of the project I am proposing is to improve the prediction model used in Thailand's index insurance program. Below, I describe several  projects that give me the experience necessary to succeed in this project. 
% In the first project, I worked on a web application that implemented a large scale randomized control trial in Haryana, India. In the second project, I used a deep learning model to predict job attributes based on textual descriptions of jobs. In the last project, I developed a method to design index insurance contracts that improves their cost efficiency. 

In the fall of 2017, I worked as a software engineer for MIT’s poverty action lab in India. My role was to maintain and expand the capabilities of a web application that implemented a randomized control trial. The application sent text reminders and mobile phone recharges to incentivize families to vaccinate their children. This project helped familiarize me with some of the practical details related to the implementation of a large scale system in a developing country.

% After completing my undergraduate studies, I worked as a software engineer for MIT’s poverty action lab in India. My role was to maintain and expand the capabilities of a web application that implemented a randomized control trial. The application sent text reminders and mobile phone recharges to incentivize families to vaccinate their children. This experience exposed me to many of the frictions that come up when implementing a large scale system in a low-income country. For example, the RCT was stalled for a couple of days because some of the nurses didn't like their tablets' cases. Similarly, there was always pressure to fix malfunctions in the application immediately, because the application also managed payments for the nurses, which meant that nurses could refuse to work if the application wasn't working. Finally, I also had many communications with the cell service provider we used to send text reminders. This project helped familiarize me with some of the practical details related to the implementation of such a system. 

% After completing my undergraduate studies, I worked as a software engineer for MIT’s poverty action lab in India. My role was to maintain and expand the capabilities of a web application that implemented a randomized control trial that aimed to find the most cost-effective way to increase the immunization rate of children in rural villages. Nurses would use tablets to upload the information of vaccinated children to a database. The application would then automatically send the text reminders and mobile phone recharges to the selected families. First, I created a website for updating our database based on the delivery status of the text reminders we sent. After that, I expanded and optimized the application’s test suite. I established a system that monitored the state of the database and sent out email alerts if the database was down. Additionally, I changed the reminder system so that multiple reminders were sent if a child had not received the vaccine he/she was being reminded for and so that a larger number of families received a text reminder. This experience exposed me to many of the frictions that come up when implementing a large scale system in a developing country.

I worked on multiple projects applying machine learning techniques to problems in economics. For my undergraduate thesis, I evaluated the performance of different machine learning algorithms on the task of classifying economics research articles. I was able to improve the system's accuracy to $92\%$ from a baseline of $78\%$. For my master's thesis, I used a deep learning model to predict job complexity based on a job's textual description. The trained model achieved an accuracy of $82\%$ and was able to predict job complexity for jobs in later years. In addition to my project experience, I have also taken graduate level machine learning classes at UChicago and MIT. 

% For my master's thesis, I used natural language processing techniques to augment the 1940 US Census with more detailed information about the work done by each person in the labor force. The goal of this project was to study the nature of new work in 1940. I used a BERT (Bidirectional Encoding Representations from Transformers) model to predict each person's job complexity based on the job's textual description in the 1939 Dictionary of Occupational Titles (DOT). The trained model achieved an accuracy of $82\%$ and was able to predict job complexity for jobs in later years. 

One of my current projects studies the design of agricultural index insurance contracts. In this project, I develop a method for simultaneously designing the insurance contracts for all insured zones. This is more cost effective because it takes into account how the correlations between zones affect the cost of the insurance. I evaluated my method against the baseline method using real and synthetic data, and found that it is more cost effective.

Through these projects, I have gained experience in working on software that is used in the field, and many of the practical considerations  necessitated by an applied setting. I have also gained experience implementing and refining deep learning models for complicated prediction tasks. Finally, through my research on agricultural index insurance, I have learned about the context and design of such programs. I believe all of these experiences will allow me to effectively work on the project I am proposing. 
 


\end{document}